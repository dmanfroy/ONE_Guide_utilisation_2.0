\chapter{Accueil extrascolaire type 2}
\section*{Introduction}
En développant l’onglet Accueil Extrascolaire, vous pouvez choisir les items suivants : Lieux, activités de vacances, agréments, subsides, présences AES2, prestation AES2 et contact. 
Les données qui font partie du rapport annuel restent modifiables jusqu’à la date de dépôt du rapport. Ceci concerne par exemple les données salariales du personnel. 

\section{Lieux}
Cet écran vous permet de consulter la liste de vos lieux extrascolaires. 
Si vous disposez de lieux AES2 et de lieux AES1, ils seront tous repris dans cette liste avec leur identifiant spécifique en commençant pas les lieux AES2. Vous pouvez choisir grâce au filtre « Type de lieux d’accueil extrascolaire » d’afficher soit la totalité des lieux AES2 et AES1, soit uniquement les lieux AES2, soit uniquement les lieux AES1. 
\begin{info}
Il vous est possible de trier la liste de vos lieux en cliquant sur le nom de la colonne via laquelle vous souhaitez effectuer le tri (par numéro, suivant la dénomination ou encore la commune).

Un filtre est disponible en haut à droite afin d’effectuer une recherche sur n’importe quel mot ou groupe de mots.
\end{info}

En cliquant sur \fbox{un de vos lieux AES2}, vous arrivez sur sa page spécifique et 3 onglets sont disponibles.

\subsection{L'onglet Lieu}
Cet onglet vous permet de consulter la dénomination et l’adresse du lieu concerné. 

\begin{remarque}
Il ne vous est pas possible de modifier ces informations, liées directement à l’agrément dont vous bénéficiez.

En cas de modification des données d’un lieu d’accueil, vous devez obligatoirement passer par votre gestionnaire de dossier qui analysera la situation et encodera les modifications ou vous informera de la procédure à suivre, par exemple lorsqu’une modification de l’agrément est requise.
\end{remarque}

Vous avez la possibilité de renseigner un Responsable de projet spécifique au lieu. 
\begin{enumerate}
    \item Première situation : la personne à encoder existe déjà dans la liste de vos contacts. Dans ce cas, il vous suffit de la sélectionner dans la liste déroulante de la rubrique « NOM Prénom ».
    \item Deuxième situation : la personne n’apparaît pas dans la liste et vous devez introduire ses coordonnées dans l’écran qui s’ouvre lorsque vous cliquez sur le bouton \ovalbox{Créer une personne}.  
\end{enumerate}

\subsection{L'onglet Horaires}
Cet onglet vous permet de nous renseigner le nombre de jours d’ouverture en période scolaire, les heures d’ouverture de votre lieu ainsi que la date à partir de laquelle ces données sont valides.

\begin{info}
Il ne vous est pas possible de déclarer un nombre de jours d’ouverture en période scolaire supérieur à 195. Pour information, une année scolaire comporte entre 180 et 185 jours. La limite de 195 jours a été fixée pour éviter au maximum les erreurs d’encodage.
\end{info}

Après avoir activé la fonction d’édition, vous devez cocher une case pour « activer » le matin ou l’après-midi d’un jour de la semaine et ensuite encoder la plage horaire.
Vous devez encoder l’offre d’accueil telle qu’elle est présentée aux parents, c’est-à-dire les heures d’accessibilité à l’accueil mentionnées dans votre ROI ou dans tout autre document à disposition des parents.

N’oubliez pas de valider les données encodées en cliquant sur le bouton bleu \ovalbox{valider} en bas à droite de l’écran. Vous ne devez pas ré encoder les horaires pour chaque année scolaire. S’il n’y a pas de modification, le système reprendra les horaires de l’année précédente. Par contre, si vos horaires changent, vous devez impérativement les mettre à jour et préciser la date de prise d’effet des nouveaux horaires. Votre gestionnaire analysera la modification et reviendra éventuellement vers vous, par exemple si les nouveaux horaires ne respectent pas les conditions de subvention.  

\subsection{L'onglet Contact ONE}
Cet onglet vous présente les différents travailleurs de l’ONE et leurs coordonnées. Ils sont les référents pour ce lieu d’accueil en ce qui concerne :
\begin{itemize}
    \item La gestion administrative = votre gestionnaire de dossier
    \item La gestion financière = votre inspecteur comptable
    \item La mise en œuvre du projet d’accueil et le PAQ = votre coordinateur accueil (ou votre conseillère EDD si le lieu est également reconnu EDD)
\end{itemize}

\section{Activités de vacances}
Cet écran vous permet d’encoder les activités de vacances et de consulter les activités que vous avez déjà renseignées, ainsi que de disposer d’un aperçu du nombre total de jours valorisables dans le cadre du subside AES2 pour les activités de vacances déjà introduites.

\textbf{Jours valorisables} : tous les jours du lundi au vendredi, en ce compris les jours fériés qui tombent pendant une semaine d’activité. Sont donc exclus les samedis et les dimanches, même si vous avez accueilli des enfants ces jours-là.

\subsection{Vous ne disposez pas de l’agrément Centre de vacances} \label{aes2-nocdv}
Toutes les activités sont rassemblées dans un cadre unique.
\begin{info}
Il vous est possible de trier la liste des activités en cliquant sur le nom de la colonne via laquelle vous souhaitez effectuer le tri.
Un filtre est disponible en haut à droite afin d’effectuer une recherche sur n’importe quel mot ou groupe de mots. 
\end{info}


\subsection{Déclaration d'une activité de vacances}

Vous déclarez une nouvelle activité de vacances en cliquant sur le bouton prévu à cet effet, en haut à droite de votre écran.

Commencez par déclarer si l’activité est résidentielle ou non. Par activité résidentielle, nous entendons que les enfants ont dormi sur le lieu de l’accueil.

Si l’activité est non résidentielle, nous vous demandons de nous indiquer les horaires d’ouverture.

Sélectionnez ensuite le type de congé (Détente, Printemps, Eté, Hiver, Automne) ainsi que les dates de début et de fin de l’accueil.
Nous vous invitons à encoder une seule activité de vacances par lieu et par type de congés. Vous avez la possibilité, via le bouton +, d’ajouter des périodes supplémentaires et, via le bouton -, de supprimer des périodes si besoin.

\begin{info}
Un outil calendrier est à votre disposition pour sélectionner facilement les dates concernées. 
\end{info}


Les week-ends n’étant pas pris en compte dans le calcul automatique de votre nombre de jours d’ouverture, vous ne devez pas craindre de les englober dans vos périodes. Par exemple, si vous organisez une activité du 1er au 31 juillet, vous la déclarez en une fois et non semaine par semaine. 

Pour chaque type de congé, vous ne pouvez choisir les dates que parmi une sélection restreinte. Par exemple, pour les vacances de Détente, vous ne pouvez sélectionner que des dates allant du 1er février au 31 mars.

\begin{attention}
Cela signifie surtout qu’une fois la date butoir dépassée, il ne vous sera plus possible d’encoder une activité de vacances pour cette période. Veillez donc bien à déclarer vos activités de vacances dès que possible mais impérativement avant qu’elles n’aient lieu.
\end{attention}

Nous vous demandons également de nous renseigner le lieu d’accueil sur lequel l’activité a été organisée.
En sélectionnant un lieu dans votre liste, l’adresse est complétée par défaut.

\begin{remarque} Il vous est toujours possible de modifier l’adresse pour les cas où l’activité de vacances aurait été organisée sur un lieu différent, non repris dans votre liste de lieux AES2, mais avec les enfants et le personnel de ce lieu.
\end{remarque}

Enfin, n’oubliez pas d’enregistrer votre déclaration via le bouton qui se trouve en bas à droite du formulaire.

\subsection{Vous disposez d’un agrément Centre de vacances}

Les activités sont présentées en deux catégories selon les périodes concernées : 
\begin{enumerate}
    \item les activités des vacances d’automne et de détente sont reprises dans un premier cadre;
    \item les activités des vacances d’hiver, de printemps et d’été dans un second cadre.
\end{enumerate}
 

\begin{information}
Cette distinction découle de la différence de procédure pour déclarer les activités qui relèvent de l’agrément Centre de vacances.
\end{information}

\begin{info}
Dans chacun des cadres séparément, il vous est possible :
\begin{itemize}
    \item De choisir l’année pour laquelle vous souhaitez consulter les activités (la sélection s’appliquera d’office aux deux cadres).
    \item De trier la liste des activités en cliquant sur le nom de la colonne via laquelle vous souhaitez effectuer le tri.
    \item D’effectuer une recherche sur n’importe quel mot ou groupe de mots à l’aide du filtre disponible en haut à droite.
    \item De déclarer une nouvelle activité de vacances en cliquant sur le bouton prévu à cet effet, en haut à droite de votre écran. 
\end{itemize}
\end{info}



\underline{\textbf{Déclarer une activité pour les vacances d’automne ou de détente}}

La procédure et les informations demandées sont identiques à ce qui est applicable aux opérateurs qui ne disposent pas de l’agrément Centre de vacances (voir point \ref{aes2-nocdv}).


\underline{\textbf{Déclarer une activité pour les vacances d’hiver, de printemps ou d’été}}

Lorsque vous cliquez sur le bouton « déclarer une activité de vacances », c’est un formulaire de déclaration d’activités Centre de vacances qui s’ouvre. Vous devez compléter toutes les informations demandées et enregistrer la déclaration via le bouton qui se trouve en bas à droite du formulaire.

\begin{remarque}
Si un opérateur AES2 dispose également d’un agrément Centre de vacances, nous considérons que l’ensemble des activités déclarées pour les vacances d’hiver, de printemps et d’été relèvent des deux agréments AES et CDV. Dès lors, la case « subventionnée dans le cadre AES2 » est cochée par défaut et vous ne pouvez pas la modifier. 
Si, pour ces périodes, vous souhaitez déclarer une activité qui ne dépend pas de l’AES2 (par exemple sur un site non repris dans la liste des lieux agréés AES), vous devez le faire via la rubrique Centre de vacances dans le menu à gauche de l’écran d’accueil ou la touche de raccourci au niveau des Centres de vacances. Par ce biais, vous pourrez remplir une déclaration d’activités en dehors du cadre des subventions AES2 et introduire par la suite une demande de subside Centre de vacances pour cette activité.
\end{remarque}

Les déclarations relatives à des activités Centre de vacances subventionnées dans le cadre AES2 sont visibles à la fois dans la liste des activités de vacances du menu Accueil extrascolaire et dans la liste des activités du menu Centre de vacances.

Lorsque vous avez validé la création d’une activité de vacances, via l’une ou l’autre procédure, les données encodées ne sont plus éditables par les opérateurs. Pour toute modification, il convient de contacter votre gestionnaire de dossier. Il pourra

\begin{itemize}
    \item Corriger une information relative à une activité vacances ;
    \item Encoder le statut « annulé » lorsqu’une activité programmée n’a pas eu lieu, par exemple par manque de participants ou parce que les locaux n’étaient pas disponibles. Dans ce cas, l’activité restera visible mais ne sera pas prise en compte dans l’analyse du dossier. Cette fonctionnalité est effective actuellement pour les déclarations d’activités liées à l’agrément CDV mais sera étendue à l’ensemble des activités de vacances ;
    \item Supprimer totalement une activité, par exemple si vous avez créé deux fois la même activité. 
\end{itemize}

\section{Agréments}

Vous retrouvez ici la liste des communes sur le territoire desquelles vous organisez un accueil.
Cette liste mentionne également :

\begin{itemize}
    \item Si la commune dispose d’un programme CLE ou non ;
    \item La période d’agrément pour chaque commune ;
    \item Le nombre de vos lieux par commune.
\end{itemize}

\begin{information}
La procédure de renouvellement d’agrément des programmes CLE étant relativement longue, la décision intervient souvent plusieurs mois après l’échéance du cycle précèdent. De plus, l’actualisation de ces informations se fait encore manuellement. Dès lors, ne vous inquiétez pas si la date de fin d’agrément renseignée à l’écran est dépassée.
\end{information}

\section{Subsides}\label{aes2_subsides}
\begin{remarque} si vous disposez à la fois de lieux AES1 et de lieux AES2, cette entrée ouvre par défaut l’écran d’accueil des subsides AES1. Pour accéder aux données de l’AES2, il suffit de sélectionner l’onglet AES2 en haut à gauche de l’écran.
\end{remarque}

Pour l’AES2, vous trouverez dans les différents écrans accessibles depuis cette entrée :
\begin{itemize}
    \item Une majorité de données uniquement consultables par le PO et communiquées à titre informatif ;
    \item L’écran dans lequel vous allez encoder les frais de fonctionnement et frais annexes à l’engagement des travailleurs, données qui font partie du rapport annuel ;
    \item Le bouton de validation définitive de votre encodage qui clôture l’année concernée et acte la finalisation du rapport annuel.    
\end{itemize}

Lorsque l’on ouvre le volet \ovalbox{subside AES2}, un premier écran apparaît reprenant la liste des subsides initiaux par année. On y trouve les données suivantes :

\begin{itemize}
    \item L’année concernée par le subside (avec possibilité de trier)
    \item Le budget global : il s’agit du budget prévisionnel en début d’année sur base duquel les avances sont calculées. travailleurs, données qui font partie du rapport annuel ;
    \item Le bouton de validation définitive de votre encodage qui clôture l’année concernée et acte la finalisation du rapport annuel.    
    \item La capacité subsidiable AES (et le cas échéant la capacité subsidiable flexible)   
    \item La date de création   
    \item La date de mise à jour si une modification est intervenue    
\end{itemize}

En cliquant sur la ligne du subside choisi, vous ouvrez un nouvel écran qui présente une première rubrique « données du subside » et plus bas quatre onglets détaillant les « avances trimestrielles », les « données réelles », les frais réels et le « subside actualisé ».
\subsection{Données du subsides}

Il s’agit toujours des données de base qui ont servi au calcul du budget prévisionnel en début d’année. Les rubriques relatives à l’accueil flexible sont présentes uniquement pour les opérateurs qui sont subsidiés pour ce type d’accueil.

\begin{itemize}
    \item \textbf{Année}
    \item \textbf{Capacité subsidiable} : ne varie pas (sauf exceptions pour lesquelles vous recevez une notification officielle)
    \item Capacité subsidiable flexible : ne varie pas (sauf exceptions pour lesquelles vous recevez une notification officielle)
    \item \textbf{Présences subsidiables « AES »} : total des présences que vous avez encodées dans les tableaux récapitulatifs « présences AES2 ». Ce chiffre évolue au fur et à mesure de votre encodage (voir point \ref{aes2_récapitulatif}) et permet de vous situer en cours d’année par rapport à la capacité subsidiable
    \item \textbf{Présences subsidiables « flexible »} : total des présences que vous avez encodées dans les tableaux récapitulatifs « présences flexibles ». Ce chiffre évolue au fur et à mesure de votre encodage (voir point \ref{aes2_récapitulatif}) et permet de vous situer en cours d’année par rapport à la capacité subsidiable
    \item \textbf{Anciennetés moyennes Responsables et Encadrants} : ne varient pas, basées sur les données de l’exercice précédent 
    \item \textbf{Autres subventions} : ce sont les cofinancements pris en compte dans le calcul de l’enveloppe et indexés chaque année. Ils ne varient pas en cours d’exercice et peuvent être différents des cofinancements réels
    \item \textbf{Budget estimé} : ne varie pas, subside prévu en début d’année servant de base au calcul des avances trimestrielles
\end{itemize}

\subsection{Avances trimestrielles}
Reprend le pourcentage par rapport au subside total (budget estimé) et les montants des 4 avances prévues en cours d’année.  

Indique également la date de création du paiement de l’avance lorsque celle-ci a été réalisée. Notez qu’en raison du délai de traitement des fichiers de paiement, il y aura systématiquement un décalage de quelques jours entre la date de création qui apparaît à l’écran et la date effective du paiement.

\subsection{Données réelles}\label{aes2_données_réelles}
Cet écran vous donne un aperçu des données réelles de votre accueil en fonction de l’encodage effectué. Ces données évoluent donc en cours d’année.


\begin{itemize}
    \item \textbf{Tableau récapitulatif des données salariales} : reprend le total des montants encodés pour tous les travailleurs subsidiés en ce qui concerne le coût total, les cofinancements et le montant subventionnable (voir point \ref{aes2_prestation}).
    \item \textbf{Tableau récapitulatif du cadastre réel} de l’ensemble du personnel encodé dans la rubrique « prestations AES2 » (voir point \ref{aes2_prestation}). Les travailleurs sont répartis en 4 catégories : Responsables de projets, accueillants, autres (= personnel logistique/administratif ex-Fesc) et volontaires. Les catégories sont également scindées entre subsidiés et non subsidiés.
    \item \textbf{Taux d’encadrement} : 
        \begin{itemize}
            \item total des présences réelles encodées (AES2 et flexible)
            \item nombre de jours d’ouverture (encodé manuellement par votre gestionnaire sur base des horaires et activités vacances déclarées)
            \item taux d’encadrement calculé selon la règle suivante : total des présences réelles / nombre de jours d’ouverture / nombre etp accueillants subsidiés et non subsidiés
            \item ancienneté moyenne des responsables de projets et des accueillants (sur base des prestations déjà encodées). Pour le calcul de l’ancienneté moyenne, le personnel autres (= personnel logistique/administratif ex-Fesc) est repris dans les accueillants
        \end{itemize}

\end{itemize}

\begin{remarque}le taux d’encadrement exact ne peut être connu qu’en fin d’année lorsque toutes les données encodées coïncident avec le nombre de jours d’ouverture. Tant qu’il existe un décalage entre les différentes données qui interviennent dans le calcul, le résultat ne sera pas correct.
\end{remarque}


\subsection{Frais réels}
Cet écran reprend les informations qui vous étaient demandées dans l’ancien formulaire de demande de subvention. Il y a lieu de compléter dans cet onglet les différentes rubriques relatives aux :
\begin{itemize}
    \item Frais annexes à l’engagement des travailleurs
    \item Fais de fonctionnement liés à l’accueil
\end{itemize}

Les totaux s’affichent automatiquement au fur et à mesure de l’encodage.
C’est également dans cet écran que vous trouverez la case à cocher pour clôturer votre encodage. 

\begin{attention} Cette action clôture l’encodage de l’ensemble des données de l’exercice concerné et pas uniquement les informations reprises sur cet écran. Cela signifie qu’une fois cette case activée, il vous sera impossible de modifier les prestations des travailleurs ou les frais déclarés. Il s’agit donc de la dernière opération à effectuer lorsque vous avez terminé l’ensemble de l’encodage des données requises dans le dossier annuel.
\end{attention}

Pour rappel, le dossier annuel est constitué des éléments suivants, tous désormais transmis via le portail : 
\begin{itemize}
    \item Un tableau reprenant l’ensemble du personnel en place durant la période concernée : on retrouve ces informations dans la liste des prestations et dans les qualifications et formations continues de chaque travailleur ;
    \item Les justificatifs des charges salariales par travailleur : elles sont encodées dans les tableaux de données salariales des travailleurs subsidiés ;
    \item Un récapitulatif des factures : il s’agit des tableaux à compléter dans l’onglet « frais réels »
    \item Les relevés des journées de présences et le récapitulatif de l’activité : ces informations sont gérées via l’entrée « présences ».
\end{itemize}

\subsection{Subside actualisé}
Trois cadres distincts sont présents dans cet onglet.

\subsubsection{1. Cadre - Calcul du subsides}

Se retrouvent ici les informations suivantes :

\begin{itemize}
    \item Le nombre de forfaits MT par catégorie octroyé sur base de la capacité subsidiable (responsables, encadrants AES, et encadrants Flexible si d’application)
    \item Les montants des forfaits salaires correspondant aux anciennetés moyennes actualisées dans les données réelles. Il s’agit d’un montant global totalisant les forfaits des deux catégories (responsables de projets et accueillants)
    \item Le montant du forfait octroyé pour les frais de fonctionnement AES (et flexible si d’application)
    \item Les cofinancements déduits dans le calcul de l’enveloppe (différent des cofinancements réels) 
    \item Le budget actualisé revu en temps réel à partir des données encodées pour l’exercice en cours.

\end{itemize}

A la clôture d’un exercice, lorsque votre gestionnaire aura analysé et validé les données du rapport annuel, le budget actualisé deviendra le subside définitif et son montant sera figé. 


\begin{remarque}Au point \ref{aes2_subsides}, il est question des présences subsidiables alors qu’au point \ref{aes2_données_réelles}, il s’agit des présences réelles.  Les deux notions partent de la même réalité, c’est-à-dire les présences effectives à l’accueil, mais sont utilisées à des fins différentes, vérifier si la capacité subsidiable est atteinte d’un côté, et établir le taux d’encadrement de l’autre. 
\end{remarque}

Pour la majorité d’entre vous, les chiffres sont identiques, le total des présences subsidiables coïncide avec les présences réelles. La seule exception concerne les opérateurs qui disposent également d’une reconnaissance en tant qu’École de devoirs et ont bénéficié d’une subvention compensatoire en 2016. Dans ce cas, les présences subsidiables diffèrent des présences réelles car les présences EDD sont majorées de 50\%.

\subsubsection{2. Cadre - Enveloppe finale}
Le but des données présentées dans ce cadre est de vérifier si l’enveloppe est octroyée en totalité ou non. 

\begin{remarque}
L’appréciation de cette règle est basée sur la comparaison entre les présences subsidiables effectives de l’année et la capacité subsidiable théorique qui vous a été attribuée. Si l’activité subsidiable réalisée est inférieure à 90\% de la capacité subsidiable, l’enveloppe est réduite de la différence entre 90\% et la proportion réelle.  
\end{remarque}
Vous trouverez donc ici :

\begin{itemize}
    \item \textbf{Enveloppe de base} : son montant est identique au montant du subside actualisé du cadre précédent ;
    \item \textbf{Capacité subsidiable AES} : rappel de l’information déjà présente dans le cadre « données du subside » ;
    \item Présences subsidiables AES : rappel de l’information déjà présente dans le cadre « données du subside » ;
    \item \textbf{Proportion} : représente le pourcentage de réalisation de votre capacité subsidiable ;
    \item \textbf{Diminution subvention AES} : Pourcentage de diminution de la subvention AES ;
    \item \textbf{Sous-occupation AES} : Montant de la diminution de la subvention AES.
    
\end{itemize}
Lorsque l’opérateur est reconnu pour un accueil flexible, il retrouve des informations identiques pour la partie de la subvention relative à ce type d’activité. Dans ce cas, le dernier point concernant le montant de la diminution correspond au cumul des deux types d’accueil, AES et flexible. 

En cliquant sur les infobulles, vous obtenez l’explication relative aux rubriques correspondantes.

\subsubsection{3. Cadre - Justification et solde}
L’étape finale du dossier annuel est la justification de la subvention qui permet de déterminer le solde à payer ou, si nécessaire, à rembourser. 

Nous avons donc regroupé à cet endroit les totaux des charges que vous avez encodées pour le personnel subsidié et les différents types de frais. 

Cela vous permet de vérifier si ces charges sont suffisantes pour justifier la totalité de la subvention. 

\begin{attention}\normalfont
Dans le cas contraire, il vous est encore possible de modifier ou ajouter des frais éligibles pour atteindre le montant à justifier. A l’inverse, si vos frais dépassent le montant à justifier, vous pouvez en retirer.

\textcolor{rouge}{\textbf{Il est donc important de vérifier cette indication avant de valider la case de clôture définitive dans l’écran « frais réels ». }}
\end{attention}


Les différentes rubriques vous sont déjà connues car elles reproduisent les informations que vous retrouvez sur les courriers de clôture.
\begin{itemize}
    \item \textbf{Montant à justifier} : correspond à l’enveloppe de base diminuée le cas échéant du montant de la sous-occupation ;
    \item \textbf{Total des frais justifiés} : montant total retenu pour la justification de la subvention. La répartition entre les salaires, les frais annexes et les frais de fonctionnement apparaît aux rubriques suivantes ;
    \item \textbf{Subvention finale} : il s’agit du montant le moins élevé entre le « montant à justifier » et le « total des frais justifiés » ;
    \item \textbf{Avances} : reprend le montant total des avances trimestrielles payées ;
    \item \textbf{Solde à payer} : Montant final à vous payer. Si le résultat est négatif, cela signifie que vous devez le rembourser. 
\end{itemize}

En cliquant sur les infobulles, vous obtenez l’explication relative aux rubriques correspondantes.

\section{Présences AES2}
Ces écrans vont vous permettre de gérer vos remises trimestrielles de présences.
\subsection{Récapitulatif} \label{aes2_récapitulatif}

C’est ici que vous devez encoder chaque trimestre vos totaux de présences subsidiables mensuels pour tous vos lieux AES2. Ce sont ces montants qui sont utilisés pour calculer votre taux d’encadrement et vérifier que vous atteignez bien votre capacité subsidiable.

Sur la droite de l’écran, vous avez la possibilité de sélectionner l’année qui vous intéresse et le type de présences.
Si un de vos lieux a été renseigné comme EDD ou Flexible, vous aurez la possibilité d’encoder des présences de ce type pour ce lieu.
\begin{info}
L’enregistrement des données encodées se fait automatiquement lorsque vous changez de cellule ou quittez l’écran. 
\end{info}


% A encadrer 
Les présences à prendre en compte sont bien les présences subsidiables. Vous trouverez le chiffre à inscrire dans la dernière cellule de la colonne « total subsidiable » de chaque onglet du fichier Excel de recensement des présences. Pour les EDD, nous vous demandons bien d’encoder les présences majorées telles qu’elles apparaissent dans cette dernière cellule de la colonne « total subsidiable » des fichiers Excel.
%fin encadrement

\begin{attention}
Vous devez encoder les présences relatives à un trimestre \textbf{au plus tard le dernier jour du trimestre suivant}. 
\end{attention}



Trimestre par trimestre, votre gestionnaire validera les chiffres de présences renseignés dans le tableau récapitulatif. 
\begin{remarque}
Après cette étape, il ne vous sera plus possible de modifier les données relatives au(x) trimestre(s) concerné(s). Si des corrections s’avèrent nécessaires, il vous faut contacter votre gestionnaire qui pourra encore modifier les chiffres de présences jusqu’à la clôture annuelle.  
\end{remarque}


\subsection{Annexes justificatives}
Pour chaque trimestre, au plus tard le dernier jour du trimestre suivant, nous vous demandons aussi de nous fournir le fichier de présences utilisé pour calculer vos totaux de présences justifiables. Il vous suffit de joindre le fichier au trimestre concerné.

\begin{remarque}
En cas de correction des chiffres dans le tableau récapitulatif, il convient de joindre la version corrigée du fichier trimestriel de récolte des présences. Il doit toujours y avoir concordance entre les deux sources de données.
\end{remarque}

Le Guide pratique d’encodage est toujours à votre disposition dans l’onglet « ressources ». N’hésitez pas à le consulter pour toute question concernant la manière de compléter les fichiers de présences. Pour faciliter l’analyse des gestionnaires, nous vous demandons de bien respecter les consignes sur l’identification et le classement des onglets. 

\subsection{Ressources}
Cet onglet vous permet de récupérer les modèles de fichiers vierges que vous devez utiliser pour encoder vos présences chaque trimestre.
Vous avez la possibilité de télécharger : 
\begin{itemize}
    \item Le Guide pratique d’encodage qui vous indique comment renseigner vos présences ;
    \item Les fichiers AES classiques ;
    \item Les fichiers AES + Flexible ;
    \item Les fichiers EDD
\end{itemize}
Les nouveaux modèles seront systématiquement mis en ligne peu avant le début du trimestre concerné. 

\section{Prestation AES2}\label{aes2_prestation}
Les différents écrans accessibles depuis cette entrée vous permettent de consulter la liste globale du personnel ainsi que le détail de chaque travailleur. C’est également ici que vous encoderez toutes les données relatives aux formations et aux relations de travail, mais aussi aux salaires du personnel subsidié. 

\subsection{Liste des prestations AES2 }
Lorsque vous ouvrez la rubrique \ovalbox{prestations AES2}, vous accédez à une liste qui vous présente l’ensemble des travailleurs que vous avez déjà encodés via le portail (voir concernant la création d’une nouvelle prestation).

\begin{info}
\underline{Différents filtres sont disponibles au-dessus de la liste pour affiner la liste.} 
\begin{itemize}
    \item \textbf{Subventionné AES2 }: en activant ce filtre, seules les prestations cochées comme étant subventionnées AES2 apparaissent.
    \item \textbf{Actif} : en activant ce filtre, seuls les contrats de travail en cours seront visibles.
    \item \textbf{Année} : vous pouvez sélectionner l’année que vous souhaitez visualiser.
    \item \textbf{Filtre} : vous pouvez ici effectuer une recherche sur base d’un mot, d’un groupe de mots ou même d’une date. Par exemple, si vous inscrivez le nom d’un travailleur, seules les prestations de ce travailleur seront visibles. Ou encore, si vous encodez « responsable », la liste présentera uniquement les personnes renseignées comme tel.
\end{itemize}
\end{info}

\begin{info}
\underline{La liste en elle-même reprend les informations suivantes :}
\begin{itemize}
    \item \textbf{Nom Prénom}
    \item \textbf{Fonction}
    \item \textbf{Statut} : soit employé, soit volontaire.
    \item \textbf{ETP} : temps dans la fonction. Cette valeur ne peut pas dépasser le temps de travail mentionné dans le contrat encodé dans Mon Équipe. 
    \item \textbf{Période d'activité}: les dates de prestations de la personne pour l'année considérée (entre le 1er janvier et le 31 décembre). 
    \item \textbf{Cadastre}: le cadastre mentionné correspond aux salaires déjà encodés dans les données salariales. Cette donnée évolue avec votre encodage. Par exemple, un travailleur à temps plein aura un cadastre de 0,25 ETP lorsque vous aurez encodé 3 mois de salaire complet et de 0,5 ETP après 6 mois.
\end{itemize}
Vous pouvez trier chaque colonne par ordre alphabétique (ou inversé), par ordre croissant (ou décroissant) ou sans ordre défini. 
\end{info}


\subsubsection{Actions disponibles}
Plusieurs boutons sont accessibles:
\begin{itemize}[label=\textbullet]
    \item \includegraphics[width=0.3cm]{Images/icon/icon-me.png} : voir la fiche de la personne dans Mon Équipe (contrat, qualification, formations continues);
    \item \includegraphics[width=0.3cm]{Images/icon/icon-edit.png} éditer la prestation de la personne; Les données relatives à la relation de travail restent éditables, vous pouvez donc les modifier à tout moment pendant l'année à partir de l’écran des prestations AES2 d’un travailleur.
    \item \includegraphics[width=0.3cm]{Images/icon/icon-del.png} : supprimer la prestation;
    \item \includegraphics[width=0.3cm]{Images/icon/icon-subs.png} : consulter les données salariales.
    
\end{itemize}



\subsection{Créer une nouvelle prestation AES2}\label{création_prestation_aes2}
Cliquez sur le bouton \ovalbox{Nouvelle prestation}. Plusieurs informations sont demandées: 


\subsubsection{a) Identification du travailleurs}
En cliquant sur "\textbf{NOM Prénom}", vous aurez accès à la liste des personnes qui ont déjà travaillé pour votre pouvoir organisateur dans le secteur de l'AES2;  sélectionnez ensuite la personne dans la liste. Si par contre, la personne ne figure pas dans la liste, cliquez sur \ovalbox{Ajouter une personne}. Vous serez alors dirigé vers "Mon Équipe" pour créer la personne (NISS + Prénom + Nom) et son lien contractuel avec votre Pouvoir organisateur. 
\begin{conseil}
Pour vous aider, suivez le point \ref{team_add_person} "Ajouter une personne dans Mon Équipe".
\end{conseil}



\subsubsection{b) Contrat / Convention}
En cliquant sur "\textbf{Contrat et convention}", la liste vous donnera l'ensemble des liens (contractuels ou conventionnels) que la personne entretient avec votre pouvoir organisateur. La liste ne vous affichera que les contrats/conventions qui sont actifs. Si le contrat n'est pas présent dans la liste déroulante, cliquez sur \ovalbox{Nouveau contrat ou convention}. 

% DANS MON EQUIPE %

\begin{conseil}
Pour vous aider, suivez le point \ref{team_add_contract} "Ajouter un contrat/un lien". \end{conseil}    








\subsubsection{c) Fonction}
Vous avez le choix entre « \textbf{accueillant} » et « \textbf{responsable} ». Les autres membres du personnel ne doivent donc pas être renseignés.

Si la personne occupe deux fonctions avec un seul contrat de travail, vous devrez encoder une deuxième prestation et sélectionner le contrat correspondant.



\begin{info}\normalfont 
La réglementation prévoit une dérogation pour les personnes engagées dans des fonctions administratives ou logistiques aux conditions suivantes : 
\begin{itemize}
    \item Elles ont été engagées avant le 31/12/2014 ;
    \item Leur salaire était couvert en tout ou en partie par la subvention FESC en 2014.
\end{itemize}


Si ces deux conditions sont remplies, la fonction « \textbf{administratif / logistique ex-FESC} » sera disponible. Si elle n’apparaît pas, il convient de prendre contact avec votre gestionnaire de dossier. 
Théoriquement, toutes les personnes concernées par cette dérogation sont déjà encodées. Il ne devrait plus y avoir de nouveaux travailleurs répondant à ces conditions, à l’exception de situations de remplacement temporaire du titulaire du poste. Dans ce cas, le remplaçant pourrait également être subsidié. Si une telle situation se présente, il convient d’en référer à votre gestionnaire.  
\end{info}







\subsubsection{d) Statut}
Vous sélectionnez le type d’engagement du travailleur : \fbox{Employé} ou \fbox{volontaire}.

\begin{remarque} Nous avons établi cette distinction car les volontaires n’entrent pas en compte dans le calcul du taux d’encadrement. Cela signifie que tous les autres travailleurs sont globalisés dans la catégorie « employé », y compris le personnel statutaire des pouvoirs publics ou le personnel ALE.
\end{remarque}

\underline{d.1) Type d'engagement: \textbf{employé}}\\
En choisissant ce type de travailleur, une case « \textbf{Membre du personnel subventionné AES2} » apparaît. A sélectionner seulement si : \begin{itemize}
    \item Le travailleur possède un lien contractuel ou statutaire avec l’opérateur d’accueil agréé ;
    \item Son salaire est couvert, en partie ou en totalité, par la subvention AES2.
\end{itemize}

\underline{d.2) Type d'engagement: \textbf{Volontaires}}\\
Le procédé est identique à celui décrit pour un employé non-subsidié.
La différence essentielle concerne l’absence de la case « Membre du personnel subventionné AES2 », avec pour conséquence l’impossibilité d’avoir accès au tableau d’encodage des données salariales. Il n’est donc pas nécessaire de joindre les contrats des travailleurs non-subsidiés et les conventions de volontariat.

\subsubsection{e) ETP de la fonction}
Vous devez ensuite indiquer le temps de travail dans la fonction. Celui-ci peut être différent du temps de travail du contrat et doit représenter le temps presté par le travailleur dans cette fonction pour les activités AES2.




\subsubsection{f) Lieu(x) de la prestation}
Enfin, vous sélectionnez le ou les lieux d’affectation du travailleur dans la liste déroulante qui apparaît à ce niveau. Il s’agit du ou des lieux où le travailleur exerce ses fonctions habituellement. Il ne faut pas tenir compte ici d’éventuelles affectations temporaires pour pallier une absence sur un autre lieu par exemple.


\vspace{0.5cm}

Lorsque vous avez encodé toutes les informations, cliquez sur \ovalbox{créer la prestation}. 



%\input{Tables/7.aes2-consignes}


%\subsection{Encoder les données salariales de la personne}
%Pour encoder les données salariales, le contrat de la personne doit avoir un financement ONE. 


\subsection{Encoder les qualifications de la personne}
\begin{conseil}
Pour renseigner les informations de qualification de la personne, vous pouvez consulter le chapitre Mon Équipe (point \ref{sec:qualif_person}) pour vous aider.
\end{conseil}



\subsection{Encoder les formations continues d’une personne }

\begin{conseil}
Pour encoder les formations continues, référez vous au point \ref{sec:form_cont} du chapitre Mon Équipe. \textit{Cette section sera disponible dans la version 1.6 de ce Guide (édition à paraître)}.
\end{conseil}

\section{Contact}
On retrouve dans cette entrée l’adresse de la personne de contact.

Par défaut, la personne renseignée est le contact général identifié au niveau du PO. Si aucune personne n’est reprise à ce niveau, nous vous demandons de compléter les informations demandées via l’item « contact général » du menu principal de l’écran d’accueil.
Mais vous pouvez choisir de spécifier un contact propre à chacun de vos secteurs d’activité. 

Vous avez aussi la possibilité de renseigner des personnes différentes pour la gestion administrative et la gestion pédagogique.

\section{Récapitulatif des actions à effectuer et des données à encoder}
\begin{table}[H]
\resizebox{\textwidth}{!}{%
\begin{tabular}{|c|l|l|}
\hline
\rowcolor[HTML]{C0C0C0} 
\textbf{Rubrique} & \multicolumn{1}{c|}{\cellcolor[HTML]{C0C0C0}\textbf{Tâche}} & \multicolumn{1}{c|}{\cellcolor[HTML]{C0C0C0}\textbf{Calendrier}} \\ \hline
Lieu & \begin{tabular}[c]{@{}l@{}}Indiquer un responsable de projet pour chaque lieu \\ d'accueil\end{tabular} & \begin{tabular}[c]{@{}l@{}}Encodage initial ; \\ Mise à jour si modification\end{tabular} \\ \hline
Horaire & \begin{tabular}[c]{@{}l@{}}Encoder les horaires en période scolaire pour chaque \\ lieu d'accueil\end{tabular} & \begin{tabular}[c]{@{}l@{}}Encodage initial ; \\ Mise à jour si modification\end{tabular} \\ \hline
\begin{tabular}[c]{@{}c@{}}Activité de\\ vacances\end{tabular} & Déclarer vos activités de vacances & Avant le début des activités \\ \hline
Frais réels & Encoder les frais annexes et les frais de fonctionnement & \begin{tabular}[c]{@{}l@{}}Au plus tard le 31 mars de l'exercice \\ suivant\end{tabular} \\
 & Valider définitivement l'encodage des données &  \\ \hline
Récapitulatif & \begin{tabular}[c]{@{}l@{}}Encoder les totaux de présences dans le tableau \\ récapitulatif\end{tabular} & \begin{tabular}[c]{@{}l@{}}Au plus tard le dernier jour du trimestre\\ suivant\end{tabular} \\ \hline
\begin{tabular}[c]{@{}c@{}}Annexe\\ justificative\end{tabular} & Joindre le fichier de recensement des présences & \begin{tabular}[c]{@{}l@{}}Au plus tard le dernier jour du trimestre \\ suivant\end{tabular} \\ \hline
Prestation & \begin{tabular}[c]{@{}l@{}}Encoder les prestations de tous les accueillants et \\ responsables de projets\end{tabular} & \begin{tabular}[c]{@{}l@{}}Au plus tard le dernier jour du trimestre \\ suivant\end{tabular} \\
 & Joindre les justificatifs de formation de base &  \\
 & Joindre les contrats &  \\ \hline
\begin{tabular}[c]{@{}c@{}}Données\\ salariales\end{tabular} & \begin{tabular}[c]{@{}l@{}}Encoder les données salariales \\ des travailleurs subventionnés AES2\end{tabular} & \begin{tabular}[c]{@{}l@{}}Au plus tard le 31 mars de l'exercice \\ suivant\end{tabular} \\ \hline
\begin{tabular}[c]{@{}c@{}}Données\\ d'une pers.\end{tabular} & Mettre à jour les données des travailleurs & Lorsqu'il y a une modification \\ \hline
\begin{tabular}[c]{@{}c@{}}Formation\\ continue\end{tabular} & \begin{tabular}[c]{@{}l@{}}Encoder les formations continues de tous les accueillants\\  et responsables de projets\end{tabular} & \begin{tabular}[c]{@{}l@{}}Au plus tard le 31 mars de l'exercice \\ suivant\end{tabular} \\ \hline
\end{tabular}%
}
\end{table}

\begin{attention}\normalfont
Concernant l'encodage des données salariales des travailleurs subventionnées AES2, la réglementation indique le 31 mars comme date ultime pour le dépôt du rapport annuel dont font partie les données salariales. 
\end{attention}
Nous vous invitons avec insistance à ne pas attendre le dernier moment pour réaliser cet encodage et vous conseillons de l’effectuer chaque trimestre. 
Cela nous permettra d’assurer un meilleur accompagnement de la gestion administrative de votre accueil, de déceler d’éventuels problèmes et d’attirer votre attention pour vous donner le temps d’y remédier avant la clôture du rapport annuel. 

L’encodage régulier des prestations et des données salariales vous donnera également l’actualisation du montant de votre subvention sur base des données réelles de l’exercice. 
Vous pourrez ainsi constater les évolutions de la subvention en fonction de l’ancienneté moyenne du personnel subsidié et adapter vos prévisions et votre gestion en conséquence.






