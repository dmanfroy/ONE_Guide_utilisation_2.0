\chapter{Convention Type de partenariat Automne Détente}

\section{Contexte}
Le Parlement de la FW-B a adopté la réforme des rythmes scolaires en mars 2022. Les vacances d’automne (Toussaint) et de détente (Carnaval) sont, quant à elles, allongées et comptent dorénavant deux semaines au lieu d’une seule.

Afin d’élargir l’offre d’accueil durant ces périodes modifiées, différentes mesures ont été adoptées : renforcement des budgets Centres de vacances et Écoles de devoirs ainsi qu’un nouveau dispositif de subventionnement. Ce dispositif, non obligatoire, est basé sur un partenariat entre différents opérateurs. Il est intitulé PAD pour Partenariat Automne Détente. 

\section{Définitions}

\subsection{Partenariat Automne Détente (PAD)}
Un partenariat est la mise en commun de ressources dont disposent deux pouvoirs
organisateurs. Un partenariat ne peut être conclu qu’avec un service public, une ASBL ou une association de fait, ou tout autre forme juridique qui ne poursuit pas un but lucratif (les indépendants sont donc exclus de ce dispositif). 

Les ressources du partenaire « autre » (différent du partenaire agréé AES ou CDV) peuvent prendre différentes formes : 

\begin{itemize}
    \item Mise en commun de personnel pour un encadrement des enfants durant
toute la durée des activités
    \item Intervention du personnel du partenaire pour des animations spécifiques. Dans ce cas le personnel n’est pas nécessairement présent durant toute la semaine
    \item Mise à disposition de locaux adaptés
    \item Mise à disposition de matériel
    \item Prise en charge de la partie administrative (communication, inscription, facturation, organisation des activités extérieures …)
    
    Ex : un partenariat avec un CPAS permettra de toucher un public précarisé, le CPAS étant le lien avec les familles tout en n’étant pas nécessairement associé à l’organisation des activités sur le terrain.
    \item Prise en charge des repas et / ou collations
    
    Ex : un partenariat avec une entreprise à finalité sociale qui fournit les repas et informe les enfants sur la production / distribution / nutrition C’est la CCA qui évalue si le partenariat conclu respecte les différentes balises.
\end{itemize}

Les activités seront organisées par au moins un opérateur agréé en accueil extrascolaire (AES) ou en centre de vacances (CDV), en partenariat avec au moins un autre opérateur. Les activités seront accessibles aux enfants de 2,5 ans à 15 ans avec une priorité donnée aux enfants âgés de 2,5 à 6 ans et aux enfants en situation de pauvreté.

\subsection{Opérateur agréé et autre opérateur}
\textbf{Opérateur agréé }: sont concernés tous les opérateurs CDV, tous les opérateurs AES, qu’ils soient de type 1 ou de type 2.


\textbf{Autre opérateur} : il s’agit d’un service public, d’une ASBL, ou d’une association de fait, ou de tout autre forme juridique qui ne poursuit pas un but lucratif, qu’il ou elle dispose ou non d’un agrément pour ses activités habituelles (sport, culture, enfance, …). 




\section{Critères pour le partenariat}

\begin{itemize}
\item \textbf{Public} : activités accessibles aux enfants de 2,5 à 15 ans.
\item \textbf{Publics prioritaires} :
\begin{itemize}
    \item Les \textbf{enfants de 2,5 à 6 ans}. Cette donnée sera vérifiée et permettra d’ajuster la subvention accordée.
    \item Les \textbf{enfants issus de milieux défavorisés}. Cette donnée n’est pas quantifiable. Il s’agit donc d’une obligation de moyen et non de résultat. Les actions entreprises présenteront un lien avec la récarité et le partenariat devra décrire les moyens mis en œuvre pour toucher ce public dans la déclaration d’activité.
\end{itemize}
\item \textbf{Types d’activité}: ludiques, artistiques et culturelles dans le respect de l’esprit vacances et des missions, tel que défini dans le Décret Centre de Vacances\footnote{« Contribuer à l'encadrement, l'éducation et l'épanouissement des enfants pendant les périodes de congés scolaires avec notamment pour objectifs de favoriser : 1° le développement physique de l'enfant, selon ses capacités, par la pratique du sport, des jeux ou d'activités de plein air ; 2° la créativité de l'enfant, son accès et son initiation à la culture dans ses différentes dimensions, par des activités variées d'animation, d'expression, de création et de communication ; 3° l'intégration sociale de l'enfant, dans le respect des différences, dans un esprit de coopération et dans une approche multiculturelle ; 4° l'apprentissage de la citoyenneté et de la participation. »}.
\item \textbf{Horaires} : L’accueil doit couvrir au minimum une période de 7 heures par jour.
\item \textbf{Période d’accueil} : Les activités doivent couvrir au minimum une semaine (à l’exception des jours fériés ou de week-end).
\item \textbf{Si Résidentiel (camps et séjours)} : Au minimum 6 jours avec 5 nuitées, en Belgique uniquement.
\item \textbf{Participation Financière des Parents (PFP)} : les principes définis dans le Code de Qualité doivent être respectés afin de garantir l’accessibilité à toutes les familles.

Pour le résidentiel et non résidentiel, les plafonds seront ceux définis par le Gouvernement pour le montant journalier maximal de participation aux frais, qui doivent être fixés avant le début de l’année scolaire 2022-2023\footnote{art. 217 du Décret sur l’adaptation des rythmes scolaires annuels.}.

\item \textbf{Locaux} : les locaux utilisés pour les activités seront adaptés à l’âge et au nombre des enfants ainsi qu’au type d’activité proposée. Les infrastructures fixes ou mobiles offriront des garanties suffisantes d’hygiène et de sécurité aux participants.
\item \textbf{Encadrement et qualifications}

\begin{itemize}
    \item \textbf{Non résidentiel} : 1 encadrant pour 8 enfants de moins de 6 ans ou 1 encadrant pour 12 enfants de plus de 6 ans. Les animateurs qualifiés disposeront du Brevet d’Animateur en Centre de Vacances, ou disposeront de la formation de base requise en AES ou de la qualification d’animateur en Ecole De Devoirs ou encore de toute formation initiale reconnue dans un des 3 secteurs comme, par exemple, les brevets d’entraineur ou de moniteur délivrés par l’ADEPS ou les brevets délivrés par le secteur socio-culturel.
    
    La présence d’un responsable sur site est obligatoire. Ce responsable sera soit un coordinateur qualifié CDV soit un responsable de projet qualifié AES. Les normes de qualification de base doivent être respectées pour au moins 1 encadrant sur 3 pour l’ensemble des partenaires.

    \item \textbf{Résidentiel} : 1 encadrant pour 8 enfants de moins de 6 ans ou 1 encadrant pour 12 enfants de plus de 6 ans. Les animateurs qualifiés disposeront du Brevet d’Animateur en Centre de Vacances.
    
    La présence d’un responsable sur site est obligatoire. Ce responsable sera un coordinateur qualifié CDV (ou un animateur qualifié pour les camps organisés par les mouvements de jeunesse). Les normes de qualification de base doivent être respectées pour au moins 1 encadrant sur 3 pour l’ensemble des partenaires.
    

\end{itemize}
Les animateurs doivent être âgés de 16 ans au minimum. Toute personne membre de l’équipe d’encadrement doit être de bonne vie et mœurs et doit pouvoir en attester sur demande.

\item \textbf{Projet d’accueil (PA)} : L’opérateur agréé dispose de son PA de base et décrira l’activité spécifique qu’il organise en partenariat durant la période concernée. Il expliquera dans la convention établie avec son / ses partenaire(s) les modalités d’organisation de l’activité, dont les moyens mis en œuvre pour toucher et accueillir les publics visés.

\item \textbf{Fiches d’inscription} : elles doivent être établies, dans le respect du RGPD, reprenant les coordonnées de l’enfant, les personnes qui l’ont confié, qui sont autorisées à venir le chercher, qui sont à joindre en cas d’urgence. Le NISS sera demandé afin de compléter les attestations fiscales uniquement.

\item \textbf{Fiches de santé} : elles doivent être complétées pour chaque enfant. Un exemple de fiche est disponible dans la brochure « Mômes en santé ».
\item \textbf{Référent santé} : une personne référente en matière de santé sera désignée et appliquera les recommandations décrites dans la brochure « Mômes en santé ».

Cette personne sera joignable durant toute la période d’activité

\end{itemize}



